%%%%%%%%%%%%%%%%%%%%%%%%%%%%%%%%%%%%%%%%%
% "ModernCV" CV and Cover Letter
% LaTeX Template
% Version 1.1 (9/12/12)
%
% This template has been downloaded from:
% http://www.LaTeXTemplates.com
%
% Original author:
% Xavier Danaux (xdanaux@gmail.com)
%
% License:
% CC BY-NC-SA 3.0 (http://creativecommons.org/licenses/by-nc-sa/3.0/)
%
% Important note:
% This template requires the moderncv.cls and .sty files to be in the same 
% directory as this .tex file. These files provide the resume style and themes 
% used for structuring the document.
%
%%%%%%%%%%%%%%%%%%%%%%%%%%%%%%%%%%%%%%%%%

%----------------------------------------------------------------------------------------
%	PACKAGES AND OTHER DOCUMENT CONFIGURATIONS
%----------------------------------------------------------------------------------------

\documentclass[11pt,a4paper,sans]{moderncv} % Font sizes: 10, 11, or 12; paper sizes: a4paper, letterpaper, a5paper, legalpaper, executivepaper or landscape; font families: sans or roman

\moderncvstyle{casual} % CV theme - options include: 'casual' (default), 'classic', 'oldstyle' and 'banking'
\moderncvcolor{orange} % CV color - options include: 'blue' (default), 'orange', 'green', 'red', 'purple', 'grey' and 'black'

\usepackage{lipsum} % Used for inserting dummy 'Lorem ipsum' text into the template
\usepackage[utf8]{inputenc}

\usepackage[scale=0.75]{geometry} % Reduce document margins
%\setlength{\hintscolumnwidth}{3cm} % Uncomment to change the width of the dates column
%\setlength{\makecvtitlenamewidth}{10cm} % For the 'classic' style, uncomment to adjust the width of the space allocated to your name

%----------------------------------------------------------------------------------------
%	NAME AND CONTACT INFORMATION SECTION
%----------------------------------------------------------------------------------------

\firstname{Franco} % Your first name
\familyname{Bonafé} % Your last name


%----------------------------------------------------------------------------------------

\begin{document}
\makecvtitle % Print the CV title
%----------------------------------------------------------------------------------------
%	PERSONAL DATA SECTION
%----------------------------------------------------------------------------------------
\section{Personal Information}
\cvitem{Name}{Franco Paul Bonafé}
\cvitem{Date of birth}{July 18th, 1990}
\cvitem{Birth place}{Córdoba, Argentina}
%\cvitem{ID No.}{36125009}
\cvitem{Passport No.}{AAA181677}
\cvitem{Address}{Große Brunnestraße 128, 22763 Hamburg, Germany}
\cvitem{Phone No.}{+49 152 25170313 | +54 9 351 547 2791 (WhatsApp)}
\cvitem{E-mail}{\texttt{franco.bonafe@mpsd.mpg.de | francobonafe@gmail.com}} 
\cvitem{Work address:}{Max Planck Institute for the Structure and Dynamics of Matter, Luruper Chaussee 149, 22761 Hamburg, Germany}

%----------------------------------------------------------------------------------------
%	EDUCATION SECTION
%----------------------------------------------------------------------------------------

\section{Education}

\cventry{2014--2018}{PhD in Chemical Sciences}{Dpmt. of Theoretical and Computational Chemistry, School of Chemical Sciences, UNC}{Supervisor: Prof. Cristián G. Sánchez}{thesis title: ``UV/visible photoinduced nuclear motion  in molecular and nanoscopic systems''}{Defense held on 13.12.2018}
\cventry{2009--2014}{Licenciate in Chemistry, Orientation: Physical Chemistry}{School of Chemical Sciences}{UNC, Argentina}{\textit{GPA: 9.82/10}}{}
\cventry{2003--2008}{Chemical Techician, Orientation: Food Industry}{Dr. Manuel Lucero Secondary School (C\'ordoba)}{}{\textit{GPA: 9.64/10}}{}
%----------------------------------------------------------------------------------------
%	TEACHING AND RESEARCH SECTION
%----------------------------------------------------------------------------------------

\section{Teaching and Research}


\subsection{Publications}

\cventry{2019}{``Simulation of Impulsive Vibrational Spectroscopy''}{F. J. Hernández, F. P. Bonafé, B. Aradi, Th. Frauenheim, and C. G. Sánchez}{J. Phys. Chem. A, Article ASAP}{DOI: 10.1021/acs.jpca.9b00307}{}
\cventry{2018}{``Uniform Selenization of Crack-Free Films of Cu(In,Ga)Se2 Nanocrystals''}{T. B. Harvey, F. P. Bonafé, T. Updegrave, V. Reddy Voggu, C. Thomas, S. C. Kamarajugadda, C. J. Stolle, D. Pernik, J. Du, and B. A. Korgel}{ACS App. Energy Mater., Article ASAP}{DOI: 10.1021/acsaem.8b01800}{}
\cventry{2018}{``Fully Atomistic Real-Time Simulations of Transient Absorption Spectroscopy''}{F. P. Bonafé, F. J. Hernández, B. Aradi, Th. Frauenheim, and C. G. Sánchez}{J. Phys. Chem. Lett. 9 (15) 4355}{}{}
\cventry{2017}{``Plasmon-driven sub-picosecond breathing of metal nanoparticles''}{F. P. Bonafé, B. Aradi, M. Guan, O. A. Douglas-Gallardo, C. Lian, S. Meng, Th. Frauenheim, and C. G. Sánchez}{Nanoscale 9 12391}{}{}
\cventry{2016}{``Optical Properties of Graphene Nanoflakes: Shape Matters''}{C. Mansilla Wettstein, F. P. Bonafé, M. B. Oviedo, and C. G. Sánchez}{J. Chem. Phys 144 224305}{}{}
\cventry{2015}{``Ultra-small rhenium clusters  supported on graphene''}{O. Miramontes, F.P. Bonafé, U. Santiago, E. Larios Rodríguez, J.J. Velázquez-Salazar, M. Mariscal, and M. Jose-Yacamán}{Phys. Chem. Chem. Phys. 17 7898}{}{}
\cventry{2013}{``A theoretical study of the optical properties of nanostructured TiO$_2$''}{V.C. Fuertes, C.F.A. Negre, M.B. Oviedo, F.P. Bonaf\'e, F.Y. Oliva, and C.G. S\'anchez}{J. Phys.: Cond. Matter 25 115304}{}{}

\subsection{Presentations in international scientific meetings}
\cventry{2017}{``Subpicosecond breathing mode excitation in metal nanoparticles''}{F. P. Bonafé, B. Aradi, O. A. Douglas Gallardo, Th. Frauenheim and C. G. Sánchez}{CECAM Workshop: Charge carrier dynamics in nanostructures: optoelectronics and photostimulated processes. Bremen, Germany}{Poster}{}
\cventry{2016}{``Absorption spectra and excitations from real time TD-DFTB''}{F. P. Bonafé and C. G. Sánchez}{International CECAM-Workshop \& Tutorial on Approximate Quantum Methods in the ab initio World. Beijing, China}{Tutorial}{}
\cventry{2014}{`Selenization of Automated, Ultra-Sonic Spray-Deposited Cu(In,Ga)Se$_2$ Nanocrystal Films for Photovoltaics''}{T. B. Harvey, F. P. Bonafé, T. Updegrave, C. Thomas, S. Kamarajugadda, C. J. Stolle, D. Pernik, J. Du and B. A. Korgel}{AIChE Annual Meeting. Atlanta, Georgia, USA}{Poster}{}
\cventry{2013}{``Study of the nucleation of Pd nanoparticles on graphene''}{F. P. Bonafé, G. J. Soldano, M. M. Mariscal}{XXII International Materials Research Congress (IMRC). Cancún, Mexico}{Poster}{}

\subsection{Presentations in national scientific meetings}
\cventry{2018}{``Simulations of transient absorption in time dependent DFTB''}{F. P. Bonafé, F. J. Hernández, B. Aradi, Th. Frauenheim, and C. G. Sánchez}{I Argentine Meeting of Quantum Physics, Córdoba, Argentina}{Poster}{}
\cventry{2017}{``An alternative explanation to laser-induced ultrafast vibrational excitations in metal nanoparticles''}{F. P. Bonafé, B. Aradi, O. A. Douglas-Gallardo, Th. Frauenheim, and C. G. Sánchez}{XX Argentine Meeting of Physical Chemistry and Inorganic Chemistry. Villa Carlos Paz, Córdoba, Argentina}{Talk}{}
\cventry{2017}{``Plasmonic excitation of the breating mode in metal nanoparticles''}{F. P. Bonafé, B. Aradi, S. A. Paz, O. A. Douglas-Gallardo, Th. Frauenheim, and C. G. Sánchez}{IV Nanocórdoba. Villa Carlos Paz, Córdoba, Argentina}{Talk}{}
\cventry{2015}{``Model nanomotor driven by circularly polarized light''}{F. P. Bonafé and C. G. Sánchez}{XIX Argentine Meeting of Physical Chemistry and Inorganic Chemistry. Buenos Aires, Argentina}{Talk}{}
\cventry{2013}{`TiO$_2$ as anode material for lithium-ion batteries: a computational study''}{F. P. Bonafé, F. Y. Oliva, G. L. Luque}{4th. Iberoamerican Meeting ``Hydrogen and susteinable energy sources'', Córdoba, Argentina}{Poster}{}
\cventry{2013}{``DFT and DFT+U calculations to study lithium insertion in different polymorphs of TiO${_\textrm{2}}$.''}{F. P. Bonafé, F. Y. Oliva, and G. L. Luque}{XVIII Argentine Meeting of Physical Chemistry and Inorganic Chemistry. Rosario, Argentina}{Poster}{}
\cventry{2013}{``Study of the structural parameters that influence surface reactivity of TiO${_\textrm{2}}$ nanoparticles''}{F. P. Bonafé, V. C. Fuertes, C. F. A. Negre, M. B. Oviedo, F. Y. Oliva, and C. G. Sánchez}{XVIII Argentine Meeting of Physical Chemistry and Inorganic Chemistry. Rosario, Argentina}{Poster}{}
\cventry{2011}{``Effect of the Hoffmeister series on the acid-base properties of human serum albumin: experimental study and theoretical model''}{F. P. Bonafé, O. R. Cámara, and F. Y. Oliva}{XVII Argentine Meeting of Physical Chemistry and Inorganic Chemistry. Córdoba, Argentina}{Poster}{}

\subsection{Research works in foreign institutions}
\cventry{jan-feb/2019}{Visiting PhD student}{``Development of tools, autotest suite and documentation for real-time Ehrenfest dynamics code within DFTB for release''}{BCCMS (Frauenheim Group)}{Universität Bremen, Germany}{}
\cventry{may/2018}{Visiting PhD student}{``Development of tools, autotest suite and documentation for electronic real-time TDDFTB code''}{BCCMS}{Universität Bremen, Germany}{}
\cventry{oct-dec/2017}{Visiting PhD student}{``Applications of Ehrenfest dynamics and development of a technique to compute pump-probe spectra using real-time TDDFTB''}{BCCMS}{Universität Bremen, Germany}{}
\cventry{feb-apr/2016}{Visiting PhD student}{``Implementation of electron and Ehrenfest dynamics in DFTB+''}{BCCMS}{Universität Bremen, Germany}{}
\cventry{2013}{Undergrad research work}{``Copper indium gallium selenide (CIGS) photovoltaic devices made using selenization of nanocrystal thin films''}{Korgel Lab, Department of Chemical Engineering, The University of Texas at Austin. USA}{Exchange scholarship provided by The National University of Cordoba}{4 months}

\subsection{Graduate teaching activities}

\cventry{2014--2018}{Teacher assistant}{Dpmt. of Theoretical and Computational Chemistry, School of Chemical Sciences, UNC}{Courses: Calculus I, Calculus II, Calculus III}{}{}
\cventry{2018}{Admission Course teacher}{ School of Chemical Sciences, UNC}{}{}{}

\subsection{Undergraduate teaching activities}

\cventry{2011--2014}{Undegraduate teacher assistant}{Departments of Physical Chemistry and Theoretical and Computational Chemistry, School of Chemical Sciences, UNC}{Courses: Calculus I, Calculus II, Laboratory I, Laboratory III, General Analytical Chemistry, Admission Course}{}{}
\cventry{2010--2011}{Ad-honorem teacher assistant}{Departments of Physical Chemistry and Theoretical and Computational Chemistry, School of Chemical Sciences, UNC}{}{}{}

\subsection{Preparation of course handbooks}
\cventry{2017}{``Calculus II: Handbook for Practical Classes''}{F. P. Bonafé, C. Mansilla Wettstein, C. R. Medrano, D. M. Márquez, L. Reinaudi}{School of Chemical Sciences, UNC}{}{}

% \subsection{Funding}
% \cventry{2014--2016}{Dinámica cuántica electrónica de no-equilibrio en agregados moleculares funcionalizados}{SeCyT UNC}{Director: Cristián G. Sánchez}{Monto total: \$20.000}{Función: integrante}
% \cventry{2016--2018}{Simulación de la transferencia de carga fotoinducida en celdas solares sensibilizadas por colorantes”, Subsidio otorgado por la Secretaría de Ciencia y Tecnología de la Universidad Nacional de Córdoba}{SeCyT UNC}{Director: Cristián G. Sánchez}{Monto total: \$30.000}{Función: integrante}
% \cventry{2016--2021}{Diseño y Desarrollo de Diodos Emisores de Luz (LEDs) de Nueva Generación}{Consejo de Investigaciones Científicas y Técnicas (CONICET)}{Director: Juan C. Ferrero}{Monto total: \$5.000.000}{Función: integrante}


%----------------------------------------------------------------------------------------
%	SCHOLARSHIPS SECTION
%----------------------------------------------------------------------------------------

\section{Scholarships}
\cventry{2014--2018}{Doctoral fellowship}{National Council for Science and Technology (CONICET)}{Director: Dr. Cristián Sánchez}{from 01.04.2014}{}{}
\cventry{2013}{Exchange studentship}{Programa Cuarto Centenario, The National University of Córdoba}{fall semester 2013 at the University of Texas at Austin, USA}{}{}{}
\cventry{2011--2012}{Undergraduate research scholarship del C (CIN)}{National Interuniversity Council (CIN)}{Supervisors: Dr. Fabiana Oliva, Dr. Guillermina Luque}{Topics: ``Experimental and theoretical study of insertion of alkaline metal cations TiO$_2$'' (2012) and ``Effect of the electrolyte on the charge development in proteins and its applications in protein adsorption on metallic oxides'' (2011)}{}{}


%----------------------------------------------------------------------------------------
%	DEVELOPMENT SECTION
%----------------------------------------------------------------------------------------

% \section{Desarrollo de software}
% \cventry{2016-act}{Contribuciones al paquete de simulación mecanica-cuántica DFTB+}{Módulo para dinámica eletrónica en paralelo a incluirse en la próxima release}{Desarrollo conjunto con el BCCMS en Bremen, Alemania}{\url{http://www.dftbplus.org/}}{}

%----------------------------------------------------------------------------------------
%	ENTREPRENEURSHIP SECTION
%----------------------------------------------------------------------------------------

\section{Technology transfer and entrepreneurship}
\cventry{2018}{``Empowering UK-ARG'': culture and innovation exchange}{Local coordinator}{5 day event organised in Córdoba and Buenos Aires with experts in Innovation from the University of Cambridge}{Local coordinator}{}
\cventry{2015--2018}{``Quantum Dynamics'': technology based startup}{Co-founder}{Incubated at the Business Incubator, UNC}{\url{www.quantumdynamics.io}}{}
\cventry{2016}{``Vincular Córdoba'': public-private links for innovation}{Invited panelist}{Blas Pascal University}{}{}
\cventry{2016}{Naves: national entrepreneurship competition}{finalist and best scientific startup prize, representing Quantum Dynamics}{Austral University}{Buenos Aires}{}


%----------------------------------------------------------------------------------------
%	COURSES SECTION
%----------------------------------------------------------------------------------------

\section{Courses}
\cventry{2017}{Scientific Programming Techniques Workshop}{Universidad Nacional de Tucumán, Argentina}{Grade: 10/10}{}{}
\cventry{2016}{Graduate course: ``Pedagogical foundations involved in the teaching of the Chemical Sciencs''}{UNC}{Grade: 10/10}{}{}
\cventry{2015}{Graduate course: ``Quantum Dynamics''}{Fac. Cs. Químicas, UNC.}{Calificación: 10 (diez)}{}{}
\cventry{2015}{Graduate course: ``Quantum Espresso Spring School''}{UNC}{Grade: 10/10}{}{}
\cventry{2015}{Graduate course: ``The problem of conciousness from the point of view of Phylosophy of mind and Natural Sciences''}{UNC}{Grade: Pass}{}{}
\cventry{2014}{Graduate course: ``The chemical sciences in Argentina''}{School of Chemical Sciences, UNC}{Grade: 10/10}{}{}
\cventry{2014}{Graduate course: ``Quantum mechanical methods based on the DFT. Applications to nanostructured systems.''}{UNC}{Grade: 10/10}{}{}
\cventry{2014}{Microsoft Azure for Research Training}{School of Mathematics, Physics and Astronomy, UNC}{}{}{}
\cventry{2013}{Third school of GPGPU computing for scientific applications}{San Carlos de Bariloche, Argentina}{Grade: Pass}{}{}
% \begin{itemize}
% \item CUDA básico. Pablo Ezzatti, Universidad de la República, Uruguay 
% \item A crash course on Multi-GPU computing. Massimo Bernaschi, CUDA fellow, Universidad ``La Sapienza'', Italia
% \item PyOpenCL: OpenCL in Python. Andreas Klöckner, UIUC, EEUU
% \item Medical image processing. Anders Eklund, Virginia Tech, EEUU
% \item Física computacional con GPUSs. Eduardo Bringa, UNCuyo, Argentina
% \end{itemize}}
\cventry{2013}{Courses taken at the Universidad of Texas at Austin}{fall 2013}{Austin, Texas, USA}{}{
\begin{itemize}
\item Quantum Mechanics I. Steven Weinberg. Graduate course. Grade: A.
\item Quantum Physics II. Daniel Heinzen. Undergraduate course. Grade: A.
\item Thermodynamics and Statistical Mechanics. Elaine Li. Undergraduate course. Grade: A.
\end{itemize}}
% \cventry{2013}{Cursos dictados en el Congreso HYFUSEN}{}{}{}{
% \begin{itemize}
% \item Estado del arte de las baterías de litio. Dr. J. Thomas, INIFTA.
% \item Seguridad en la producción y utilización del hidrógeno. Ing. J. L. Aprea, CNEA.
% \end{itemize}}
% \cventry{2008}{Curso de Electrónica Básica}{Academia Santo Domingo}{}{}{}

%----------------------------------------------------------------------------------------
%	EXTENSION SECTION
%----------------------------------------------------------------------------------------

\section{Institutional activities and communication of Science}
\cventry{2016--2018}{Member of the Department Council}{Dpmt. of Theoretical and Computational Chemistry, School of Chemical Sciences, UNC}{}{}{}
\cventry{2014--2018}{Member of the comission for activities with secondary schools}{School of Chemical Sciences, UNC}{}{}{}
\cventry{2014--2015}{Director of ``Pensando la Ciencia''}{project to empower scientific vocation in secondary schools}{}{}{}
\cventry{2014--2015}{Speaker at the ``Week of Science''}{School of Chemical Sciences, UNC}{}{}{}
\cventry{2013}{Speaker and Organizer at University Fair ``Cuatrociencia''}{Science and technology exhibition organized as a celebration for the 400th. anniversary of the University}{Topic: Energetic revolution for a sustainable future}{}{}

%----------------------------------------------------------------------------------------
%	LANGUAGES SECTION
%----------------------------------------------------------------------------------------

\section{Languages}
\cventry{2018}{German}{Goethe Institute Bremen and Córdoba}{A2.1}{}{}
\cventry{2007}{First Certificate in English}{University of Cambridge ESOL Examinations}{B2}{}{Grade: A}
\cventry{2006}{Preliminary English Test}{University of Cambridge ESOL Examinations}{}{}{Grade: Pass with Merit}

%----------------------------------------------------------------------------------------
%	AWARDS SECTION
%----------------------------------------------------------------------------------------
\section{Awards}
\cventry{2015}{``10 Outstanding Young People'' Award}{Cordoba Stock Exchange}{}{}{}
\cventry{2014}{Valedictorian medal}{Universidad Nacional de Córdoba}{for the best student of the all the University schools and faculty}{class 2013}{}
\cventry{2014}{Universidad Award 2013}{Diplomma with ``Mention of Honor''}{}{}{}
\cventry{2008}{Award for the academic excelence}{Roela Bank}{for the best student of secondary school}{class 2008}{}
\cventry{2008}{Academic merit medal}{Dr. Manuel Lucero Secondary School}{for the best student of the last year of the secondary school}{}{}

%----------------------------------------------------------------------------------------
%	OTHER STUFF SECTION
%----------------------------------------------------------------------------------------
% \section{Otros reconocimientos}
% \cventry{2013}{Abanderado de la Facultad de Ciencias Químicas}{Universidad Nacional de Córdoba, año 2013}{}{}{Res. 726/13 H.C.D.}
% \cventry{2013}{Primer escolta de la bandera nacional de la Universidad Nacional de Córdoba}{}{}{}{Res. Rectoral 1439/13}
% \cventry{2012}{Segundo escolta de la bandera nacional de la Facultad de Ciencias Químicas}{Universidad Nacional de Córdoba}{}{}{Res. 688/12 H.C.D.}
% \cventry{2009}{Preselección para la Olimpíada Internacional de Química 2009}{Entrenamiento teórico práctico de 2 meses en la FCEN, UBA}{Preparación para la IChO 2009 (Reino Unido)}{}{}
% \cventry{2007--2008}{Olimpíada Argentina de Química}{Universidad de Buenos Aires}{Medalla de oro en nivel 1 (2007) y nivel 2 (2008). Mejor examen regional (2007 y 2008).}{}{}
% \cventry{2006--2008}{Feria de Ciencia y Tecnología}{Agencia Córdoba Ciencia y Ministerio de Ciencia y Tecnología}{Diploma de mención especial (2006). 2do. lugar en etapa provincial (2007). Participación en instancia nacional, Ciudad Autónoma de Buenos Aires (2007)}{}{}
% \cventry{2006--2008}{Olimpíada de Matemática Argentina}{Diploma de mención especial en instancia provincial (2008). Promoción a la instancia nacional (2007)}{}{}{}


\end{document}