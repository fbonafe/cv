%%%%%%%%%%%%%%%%%%%%%%%%%%%%%%%%%%%%%%%%%
% "ModernCV" CV and Cover Letter
% LaTeX Template
% Version 1.1 (9/12/12)
%
% This template has been downloaded from:
% http://www.LaTeXTemplates.com
%
% Original author:
% Xavier Danaux (xdanaux@gmail.com)
%
% License:
% CC BY-NC-SA 3.0 (http://creativecommons.org/licenses/by-nc-sa/3.0/)
%
% Important note:
% This template requires the moderncv.cls and .sty files to be in the same 
% directory as this .tex file. These files provide the resume style and themes 
% used for structuring the document.
%
%%%%%%%%%%%%%%%%%%%%%%%%%%%%%%%%%%%%%%%%%

%----------------------------------------------------------------------------------------
%	PACKAGES AND OTHER DOCUMENT CONFIGURATIONS
%----------------------------------------------------------------------------------------

\documentclass[11pt,a4paper,sans]{moderncv} % Font sizes: 10, 11, or 12; paper sizes: a4paper, letterpaper, a5paper, legalpaper, executivepaper or landscape; font families: sans or roman

\moderncvstyle{casual} % CV theme - options include: 'casual' (default), 'classic', 'oldstyle' and 'banking'
\moderncvcolor{orange} % CV color - options include: 'blue' (default), 'orange', 'green', 'red', 'purple', 'grey' and 'black'

\usepackage{lipsum} % Used for inserting dummy 'Lorem ipsum' text into the template
\usepackage[utf8]{inputenc}

\usepackage[scale=0.75]{geometry} % Reduce document margins
%\setlength{\hintscolumnwidth}{3cm} % Uncomment to change the width of the dates column
%\setlength{\makecvtitlenamewidth}{10cm} % For the 'classic' style, uncomment to adjust the width of the space allocated to your name

%----------------------------------------------------------------------------------------
%	NAME AND CONTACT INFORMATION SECTION
%----------------------------------------------------------------------------------------

\firstname{Franco} % Your first name
\familyname{Bonafé} % Your last name


%----------------------------------------------------------------------------------------

\begin{document}
\makecvtitle % Print the CV title
%----------------------------------------------------------------------------------------
%	PERSONAL DATA SECTION
%----------------------------------------------------------------------------------------
\section{Información Personal}
\cvitem{Nombre}{Franco Paul Bonafé}
\cvitem{Fecha de nac.}{18 de Julio de 1990}
\cvitem{Lugar de nac.}{Córdoba, Argentina}
\cvitem{DNI}{36.125.009}
\cvitem{Domicilio}{24 de septiembre 1346, 6 B, 5000, Córdoba}
\cvitem{Teléfono}{0351 15 547 2791}
\cvitem{Domicilio laboral:}{Departamento de Química Teórica y Computacional, Facultad de Ciencias Químicas, UNC. INFIQC (CONICET - UNC). Medina Allende s/n, Ciudad Universitaria, 5000, Córdoba}
\cvitem{Tel. laboral:}{0351 5353 853 Int. 55190}
\cvitem{e-mail:}{\texttt{fbonafe@unc.edu.ar | francobonafe@gmail.com}} 

%----------------------------------------------------------------------------------------
%	EDUCATION SECTION
%----------------------------------------------------------------------------------------

\section{Education }

\cventry{2014--2018}{Doctor en Ciencias Químicas}{Facultad de Ciencias Químicas}{Unidad Académica: INFIQC (CONICET -- UNC)}{}{}
\cventry{2009--2014}{Licenciado en Química, Orientación: Fisicoquímica}{Facultad de Ciencias Químicas}{Universidad Nacional de Córdoba}{\textit{promedio general: 9.82/10}}{}
\cventry{2003--2008}{Bachiller Técnico en Producción de Bienes y Servicios}{Instituto Secundario Dr. Manuel Lucero (C\'ordoba)}{}{\textit{promedio general: 9.64/10}}{}

%----------------------------------------------------------------------------------------
%	TEACHING AND RESEARCH SECTION
%----------------------------------------------------------------------------------------

\section{Docencia e investigación}

\subsection{Actividades docentes como auxiliar docente}

\cventry{2015--2018}{Profesor Ayudante B, dedicación simple, por concurso}{Dpto. de Qca. Teórica y Comp., Fac. Cs. Químicas, UNC}{}{}{Res. 154/15 H.C.D.}
\cventry{2018}{Docente Guía para el Ciclo de Nivelación}{Fac. Cs. Químicas, UNC}{}{}{Res. 1323/2017 H.C.D.}
\cventry{2014--2017}{Profesor Asistente, dedicación simple, interino}{Dpto. de Qca. Teórica y Comp., Fac. Cs. Químicas, UNC}{}{}{Res. 320/15, 357/2016, 978/2016 H.C.D., 599/2017 H.C.D., 613/2017 H.C.D.}
\cventry{2014}{Profesor Asistente, dedicación simple, interino}{Dpto. de Qca. Teórica y Comp., Fac. Cs. Químicas, UNC}{}{}{Res. 967/14 H.C.D.}

\subsection{Actividades docentes como ayudante alumno}

\cventry{2013--2014}{Ayudante alumno Cat. A}{Dpto. de Qca. Teórica y Comp., FCQ, UNC}{}{}{Res. 700/13 H.C.D.}
\cventry{2013}{Ayudante alumno del Ciclo de Nivelación 2013}{Fac. Cs. Químicas, UNC}{}{}{Res. 1076/12 H.C.D.}
\cventry{2012--2013}{Ayudante alumno Cat. A}{Dpto. de Fisicoquímica, FCQ, UNC}{}{}{Res. 735/12 H.C.D.}
\cventry{2011--2012}{Ayudante alumno Cat. B}{Dpto. de Fisicoquímica, FCQ, UNC}{}{}{Res. 747/11 H.C.D}
\cventry{2010--2011}{Ayudante ad-honorem}{Dptos. de Qca. Teórica y Comp. y Fisicoquímica, FCQ, UNC}{}{}{Res. 1/11, 807/11 H.C.D.}

\subsection{Labor en docencia de grado}
\cvitem{2018}{Matemática II (jefe de trabajos prácticos)}
\cvitem{2018}{Ciclo de Nivelación FCQ (docente guía)}
\cvitem{2017}{Matemática I (jefe de trabajos prácticos)}
\cvitem{2016}{Matemática II (jefe de trabajos prácticos)}
\cvitem{2015}{Matemática III, Matemática II (jefe de trabajos prácticos)}
\cvitem{2014}{Matemática I (ayudante alumno), Matemática II (auxiliar)}
\cvitem{2013}{Ciclo de Nivelación, Laboratorio I (ayudante alumno), 2do cuatrimestre de licencia por estadía en el exterior}
\cvitem{2012}{Laboratorio III, Química Analitica General (ayudante alumno)}
\cvitem{2011}{Matemática I (ad-honorem), Química Analítica General (ayudante alumno)}
\cvitem{2011}{Laboratorio II (ad-honorem)}

\subsection{Elaboración de material didáctico}
\cventry{2017}{``Matemática II: Guía de Clases Prácticas''}{F. P. Bonafé, C. Mansilla Wettstein, C. R. Medrano, D. M. Márquez, L. Reinaudi}{FCQ, UNC}{}{}

\subsection{Publicaciones}

\cventry{2019}{``Simulations of Impulsive Vibrational Spectroscopy''}{F. J. Hernández, F. P. Bonafé, B. Aradi, Th. Frauenheim, and C. G. Sánchez}{J. Phys. Chem. A (en proceso de revisión)}{}{}
\cventry{2018}{``Uniform Selenization of Crack-Free Films of Cu(In,Ga)Se2 Nanocrystals''}{T. B. Harvey, F. P. Bonafé, T. Updegrave, V. Reddy Voggu, C. Thomas, S. C. Kamarajugadda, C. J. Stolle, D. Pernik, J. Du, and B. A. Korgel}{ACS App. Energy Mater., Article ASAP}{DOI: 10.1021/acsaem.8b01800}{}
\cventry{2018}{``Fully Atomistic Real-Time Simulations of Transient Absorption Spectroscopy''}{F. P. Bonafé, F. J. Hernández, B. Aradi, Th. Frauenheim, and C. G. Sánchez}{J. Phys. Chem. Lett. 9 (15) 4355}{}{}
\cventry{2017}{``Plasmon-driven sub-picosecond breathing of metal nanoparticles''}{F. P. Bonafé, B. Aradi, O. A. Douglas-Gallardo, C. Lian, S. Meng, Th. Frauenheim and C. G. Sánchez}{Nanoscale 9 12391}{}{}
\cventry{2016}{``Optical Properties of Graphene Nanoflakes: Shape Matters''}{C. Mansilla Wettstein, F. P. Bonafé, M. B. Oviedo and C. G. Sánchez}{J. Chem. Phys 144 224305}{}{}
\cventry{2015}{``Ultra-small rhenium clusters  supported on graphene''}{O. Miramontes, F.P. Bonafé, U. Santiago, E. Larios Rodríguez, J.J. Velázquez-Salazar, M. Mariscal, M. Jose-Yacamán}{Phys. Chem. Chem. Phys. 17 7898}{}{}
\cventry{2013}{``A theoretical study of the optical properties of nanostructured TiO$_2$''}{V.C. Fuertes, C.F.A. Negre, M.B. Oviedo, F.P. Bonaf\'e, F.Y. Oliva and C.G. S\'anchez}{J. Phys.: Cond. Matter 25 115304}{}{}

\subsection{Participación en reuniones científicas internacionales}
\cventry{2017}{``Subpicosecond breathing mode excitation in metal nanoparticles''}{F. P. Bonafé, B. Aradi, O. A. Douglas Gallardo, Th. Frauenheim and C. G. Sánchez}{CECAM Workshop: Charge carrier dynamics in nanostructures: optoelectronics and photostimulated processes. Bremen, Alemania}{Póster}{}
\cventry{2016}{``Tutorial: Absorption spectra and excitations from real time TD-DFTB''}{F. P. Bonafé, C. G. Sánchez}{International CECAM-Workshop \& Tutorial on Approximate Quantum Methods in the ab initio World. Beijing, China}{Presentación oral}{}
\cventry{2014}{`Selenization of Automated, Ultra-Sonic Spray-Deposited Cu(In,Ga)Se$_2$ Nanocrystal Films for Photovoltaics''}{T. B. Harvey, F. P. Bonafé, T. Updegrave, C. Thomas, S. Kamarajugadda, C. J. Stolle, D. Pernik, J. Du and B. A. Korgel}{AIChE Annual Meeting. Atlanta, Georgia, Estados Unidos}{Póster}{}
\cventry{2013}{``Study of the nucleation of Pd nanoparticles on graphene''}{F. P. Bonafé, G. J. Soldano, M. M. Mariscal}{XXII International Materials Research Congress (IMRC). Cancún, México}{Póster}{}

\subsection{Participación en reuniones científicas nacionales}
\cventry{2017}{``Una explicación alternativa a las excitaciones vibracionales ultrarrápidas inducidas por láser en nanopartículas metálicas''}{F. P. Bonafé, B. Aradi, O. A. Douglas-Gallardo, Th. Frauenheim y C. G. Sánchez}{XX Congreso Argentino de Fisicoquímica y Química Inorgánica. Villa Carlos Paz, Córdoba, Argentina}{Presentación oral}{}
\cventry{2017}{``Excitación plasmónica del modo de respiración en nanopartículas metálicas''}{F. P. Bonafé, B. Aradi, S. A. Paz, O. A. Douglas-Gallardo, Th. Frauenheim y Cristián G. Sánchez}{IV Nanocórdoba. Villa Carlos Paz, Córdoba, Argentina}{Presentación oral}{}
\cventry{2015}{``Nanomotor modelo impulsado por luz polarizada''}{F. P. Bonafé, C. G. Sánchez}{XIX Congreso Argentino de Fisicoquímica y Química Inorgánica. Buenos Aires, Argentina}{Presentación oral}{}
\cventry{2016}{``Absorción óptica de nanoflakes de grafeno''}{C. Mansilla Wettstein, F. P. Bonafé, C. G. Sánchez, M. B. Oviedo}{Jornadas de Posgrado, Facultad de Ciencias Químicas, Universidad Nacional de Córdoba, Argentina. Poster.}{}{}
\cventry{2015}{``Nanomotor modelo impulsado por luz polarizada''}{F. P. Bonafé, C. G. Sánchez}{XIX Congreso Argentino de Fisicoquímica y Química Inorgánica. Buenos Aires, Argentina}{Presentación oral}{}
\cventry{2013}{`TiO$_2$ como material de ánodo para baterías de ion-litio: un estudio computacional''}{F. P. Bonafé, F. Y. Oliva, G. L. Luque}{5to. Congreso nacional - 4to. Congreso iberoamericano ``Hidrógeno y fuentes sustentables de energía'' (HYFUSEN). Córdoba, Argentina}{Póster}{}
\cventry{2013}{``Estudio por cálculos DFT and DFT+U de la interacción de litio con diferentes polimorfos de TiO$_2$''}{F. P. Bonafé, F. Y. Oliva, G. L. Luque}{XVIII Congreso Argentino de Fisicoquímica y Química Inorgánica. Rosario, Argentina}{Póster}{}
\cventry{2013}{`Estudio de los parámetros estructurales que influyen en la reactividad superficial de nanopartículas de TiO$_2$''}{F. P. Bonafé, V. C. Fuertes, C. F. A. Negre, M. B. Oviedo, F. Y. Oliva, C. G. Sánchez}{XVIII Congreso Argentino de Fisicoquímica y Química Inorgánica. Rosario, Argentina}{Póster}{}
\cventry{2013}{`Efecto de la serie de Hofmeister sobre las propiedades ácido-base de la albúmina sérica humana: estudio experimental y modelo teórico''}{F. P. Bonafé, O. R. Cámara, F. Y. Oliva}{XVII Congreso Argentino de Fisicoquímica y Química Inorgánica. Córdoba, Argentina}{Póster}{}

\subsection{Trabajos de investigación}
\cventry{2017}{Estadía corta en el exterior}{``Implementación y aplicaciones de dinámica no-adiática de procesos fotoinducidos''}{Bremen Center for Computational Material Science}{Universität Bremen, Alemania}{}
\cventry{2016}{Estadía corta en el exterior}{``Implementación de dinámica electrónica y dinámica de Ehrenfest en \texttt{DFTB+}''}{Bremen Center for Computational Material Science}{Universität Bremen, Alemania}{}
\cventry{2014}{Doctorado en Ciencias Químicas}{``Relajación de excitaciones electrónicas en sistemas nanoscópicos''}{INFIQC (CONICET - UNC)}{Departamento de Qca. Teórica y Comp., FCQ, UNC}{}
\cventry{2013}{Practicanato de Licenciatura en Química, parte 2}{``Copper indium gallium selenide (CIGS) photovoltaic devices made using selenization of nanocrystal thin films''}{Department of Chemical Engineering, The University of Texas at Austin. Estados Unidos}{}{}
\cventry{2013}{Practicanato de Licenciatura en Química, parte 1}{``Estudio teórico del mecanismo de nucleación de nanopartículas de Pd sobre grafeno con aplicaciones en sensores de hidrógeno''}{Departamento de Qca. Teórica y Comp., FCQ, UNC}{}{}
\cventry{2013}{Trabajo de investigación como alumno de grado}{``Estudios experimentales y teóricos de incorporación de cationes de metales alcalinos en óxido de titanio''}{Departamentos de Fisicoquímica y de Qca. Teórica y Comp., Fac. Cs. Químicas, Universidad Nacional de Córdoba}{}{}
\cventry{2012}{Trabajo de investigación como alumno de grado}{``Efecto de la naturaleza del electrolito en el desarrollo de carga de proteínas. Su aplicación en el proceso de adsorción sobre superficies de óxidos metálicos''}{Departamento de Fisicoquímica, Fac. Cs. Químicas, Universidad Nacional de Córdoba}{}{}

\subsection{Financiamiento}
\cventry{2014--2016}{Dinámica cuántica electrónica de no-equilibrio en agregados moleculares funcionalizados}{SeCyT UNC}{Director: Cristián G. Sánchez}{Monto total: \$20.000}{Función: integrante}
\cventry{2016--2018}{Simulación de la transferencia de carga fotoinducida en celdas solares sensibilizadas por colorantes”, Subsidio otorgado por la Secretaría de Ciencia y Tecnología de la Universidad Nacional de Córdoba}{SeCyT UNC}{Director: Cristián G. Sánchez}{Monto total: \$30.000}{Función: integrante}
\cventry{2016--2021}{Diseño y Desarrollo de Diodos Emisores de Luz (LEDs) de Nueva Generación}{Consejo de Investigaciones Científicas y Técnicas (CONICET)}{Director: Juan C. Ferrero}{Monto total: \$5.000.000}{Función: integrante}


%----------------------------------------------------------------------------------------
%	SCHOLARSHIPS SECTION
%----------------------------------------------------------------------------------------

\section{Becas obtenidas}
\cventry{2013}{Beca interna doctoral del Consejo Nacional de Investigaciones Científicas y Técnicas por el término de 60 meses a partir del 1 de abril de 2014}{Director: Dr. Cristián Sánchez}{Puntaje: 96,30/100}{}{Res. D. 4830/2013 CONICET}
\cventry{2012}{Beca de movilidad estudiantil Programa Cuarto Centenario}{Exención de matrícula y ayuda económica. Prosecretaría de Relaciones Internacionales, UNC}{2do cuatrimestre de 2013 cursado en la Universidad de Texas en Austin, Estados Unidos}{}{}{}
\cventry{2012}{Beca de estímulo a las vocaciones científicas del Consejo Interuniversitario Nacional (CIN)}{Directoras: Dra. Fabiana Oliva, Dra. Guillermina Luque}{Puntaje: 98.80/100}{}{Res. Rectoral 2724/12}
\cventry{2011}{Beca de estímulo a las vocaciones científicas del Consejo Interuniversitario Nacional (CIN)}{Directora: Dra. Fabiana Oliva}{Puntaje: 98.92/100}{}{Res. Rectoral 2382/11}


%----------------------------------------------------------------------------------------
%	DEVELOPMENT SECTION
%----------------------------------------------------------------------------------------

\section{Desarrollo de software}
\cventry{2016-act}{Contribuciones al paquete de simulación mecanica-cuántica DFTB+}{Módulo para dinámica eletrónica en paralelo a incluirse en la próxima release	}{Desarrollo conjunto con el BCCMS en Bremen, Alemania}{\url{http://www.dftbplus.org/}}{}

%----------------------------------------------------------------------------------------
%	ENTREPRENEURSHIP SECTION
%----------------------------------------------------------------------------------------

\section{Emprendedorismo y transferencia tecnológica}
\cventry{2016}{Panelista invitado en ``Vincular Córdoba''}{jornadas de articulación público-privadas para la innovación}{Universidad Blas Pascal}{}{}
\cventry{2016}{Finalista y mención a mejor emprendimiento científico en competencia nacional de emprendimientos Naves}{en representación de Quantum Dynamics}{IAE - Universidad Austral}{}{}
\cventry{2015}{Emprendedor en el proyecto ``Quantum Dynamics'' en la Incubadora de Empresas de la Universidad Nacional de Córdoba}{anteriormente ``Óptica in sílico''}{Parque Científico Tecnológico de la UNC}{\url{www.quantumdynamics.io}}{Res. SECyT UNC 390/14}
\cventry{2011}{``Emprendedor E+E''}{Fundación Empresarial para Emprendedores}{aprobación del seminario de planes de negocios}{}{}{}
\cventry{2010}{Primer puesto en concurso de planes de negocios}{Programa Emprendedores Tecnológicos 2010}{Junior Achievement, Motorola y Banco Galicia}{}{}{}

% %   \item Seminario Emprendedores Tecnológicos, organizado por Junior Achievement, Motorola y Banco Galicia. Córdoba, 8/05/2010.


%----------------------------------------------------------------------------------------
%	COURSES SECTION
%----------------------------------------------------------------------------------------

\section{Cursos}
\cventry{2017}{Workshop en Técnicas de Programación Científica}{Universidad Nacional de Tucumán, Argentina}{Calificación: 10 (diez)}{}{}
\cventry{2016}{Curso de posgrado: ``Fundamentos pedagógicos y didácticos en la enseñanza de las ciencias químicas''}{Fac. Cs. Químicas, UNC.}{Calificación: 10 (diez)}{}{}
\cventry{2015}{Curso de posgrado: ``El problema de la conciencia desde el punto de vista de las filosofías de la mente y de las ciencias naturales''}{Fac. Cs. Químicas, UNC.}{Calificación: 10 (diez)}{}{}
\cventry{2015}{Curso de posgrado: ``Dinámica cuántica''}{Fac. Cs. Químicas, UNC.}{Calificación: 10 (diez)}{}{}
\cventry{2015}{Curso de posgrado: ``Quantum Espresso Spring School''}{Fac. Cs. Químicas, UNC.}{Calificación: 10 (diez).}{}{}
\cventry{2014}{Curso de posgrado: ``La problemática de las ciencias químicas en Argentina''}{Fac. Cs. Químicas, UNC.}{Calificación: aprobado}{}{}
\cventry{2014}{Curso de posgrado: ``Métodos mecanocuánticos basados en la DFT. Aplicaciones a sistemas nanoestructurados''}{Fac. Cs. Químicas, UNC}{Calificación: 10 (diez).}{}{}
\cventry{2014}{Microsoft Azure for Research Training}{FaMAF, UNC}{}{}{}
\cventry{2013}{3ra Escuela Argentina de GPGPU Computing para aplicaciones científicas}{San Carlos de Bariloche, Argentina}{Examen aprobado}{}{}
% \begin{itemize}
% \item CUDA básico. Pablo Ezzatti, Universidad de la República, Uruguay 
% \item A crash course on Multi-GPU computing. Massimo Bernaschi, CUDA fellow, Universidad ``La Sapienza'', Italia
% \item PyOpenCL: OpenCL in Python. Andreas Klöckner, UIUC, EEUU
% \item Medical image processing. Anders Eklund, Virginia Tech, EEUU
% \item Física computacional con GPUSs. Eduardo Bringa, UNCuyo, Argentina
% \end{itemize}}
\cventry{2013}{Clases tomadas en la Universidad de Texas en Austin}{Fall 2013}{ Austin, Texas, Estados Unidos}{}{
\begin{itemize}
\item Quantum Mechanics I. Steven Weinberg. Curso de posgrado. Calificación: A.
\item Quantum Physics II. Daniel Heinzen. Curso de grado. Calificación: A.
\item Thermodynamics and Statistical Mechanics. Elaine Li. Curso de grado. Calificación: A.
\end{itemize}}
% \cventry{2013}{Cursos dictados en el Congreso HYFUSEN}{}{}{}{
% \begin{itemize}
% \item Estado del arte de las baterías de litio. Dr. J. Thomas, INIFTA.
% \item Seguridad en la producción y utilización del hidrógeno. Ing. J. L. Aprea, CNEA.
% \end{itemize}}
\cventry{2008}{Curso de Electrónica Básica}{Academia Santo Domingo}{}{}{}

%----------------------------------------------------------------------------------------
%	EXTENSION SECTION
%----------------------------------------------------------------------------------------

\section{Actividades institucionales, de extensión y articulación}
\cventry{2016--2018}{Miembro suplente del claustro de profesores auxiliares en el Consejo Departamental}{DQTC, FCQ, UNC}{}{}{Res. 888/2016 H.C.D}
\cventry{2014--2018}{Miembro titular de la comisión de Articulación con Escuelas Secundarias de la Facultad de Ciencias Químicas}{}{}{}{Res. 1040/14 H.C.D}
\cventry{2017}{Co-organizador de experimento interactivo ``Rocket Science'' en el marco del evento ``Shape Latam''}{declarado de interés cultural por el Ministerio de Cultura de la Nacion}{7-11 abril, Córdoba}{}{}
\cventry{2016}{Miembro del comité organizador del Congreso de Ciencia, Universidad y Sociedad ``Ciencia crítica y crítica de la ciencia''}{avalado por FCQ-UNC, FaMAF-UNC, FFyH-UNC, FCS-UNVM, CCT-CONICET Córdoba y Min. de Ciencia y Tecnología de la Prov. de Córdoba}{11-13 noviembre, Córdoba}{}{Res. 946/2016 H.C.D}
\cventry{2015}{Director de proyecto de articulación con escuelas secundarias}{``Pensando la Ciencia''}{}{}{Res. 374/15 H.C.D}
\cventry{2015}{Coordinador departamental y a cargo de la exposición ``charla del científico'' en la actividad de articulación}{``Semana de la Ciencia 2015''}{Departamento de Qca. Teórica y Comp., FCQ, UNC}{}{Res. 973/15 H.C.D}
\cventry{2014}{Expositor}{``Semana de la Ciencia 2014''}{Módulo: ``Simuladores al rescate''}{}{Res. 923/14 H.C.D}
\cventry{2013}{Miembro del comité evaluador de carrera docente}{Observador estudiantil}{FCQ, UNC}{}{Res. 129/13 H.C.D}
\cventry{2013}{Becario expositor en Cuatrociencia}{Muestra de Ciencia, Arte y Tecnología de la UNC}{Stand: ``Revolución energética para un futuro sustentable''}{}{Res. Rectoral 483/13}
\cventry{2013}{Participación en calidad de organizador del stand ``Revolución energética para un futuro sustentable''}{Muestra Cuatrociencia}{Fabricación de un aparato de electrólisis y póster}{}{Res. 497/13 H.C.D}
\cventry{2012}{Orador en muestra de carreras ``La UNC te espera''}{Facultad de Ciencias Químicas, UNC}{}{}{Res. 886/12 H.C.D}
\cventry{2008--2013}{A cargo del entrenamiento de los alumnos del Instituto Dr. Manuel Lucero para la Olimpíada Argentina de Química}{}{}{}{}

%----------------------------------------------------------------------------------------
%	LANGUAGES SECTION
%----------------------------------------------------------------------------------------

  \section{Conocimiento de idiomas}
\cventry{2018}{Alemán Nivel A2.1}{Goethe Institut Córdoba}{}{}{}
\cventry{2007}{First Certificate in English}{University of Cambridge ESOL Examinations}{B2}{}{Grade: A}
\cventry{2006}{Preliminary English Test}{University of Cambridge ESOL Examinations}{}{}{Grade: Pass with Merit}


%----------------------------------------------------------------------------------------
%	AWARDS SECTION
%----------------------------------------------------------------------------------------
\section{Premios}
\cventry{2015}{Premio ``10 Jóvenes Sobresalientes''}{Bolsa de Comercio de Córdoba}{}{}{}
\cventry{2014}{Medalla al Mejor Promedio}{Universidad Nacional de Córdoba}{Promoción 2013}{}{}
\cventry{2014}{Premio Universidad 2013}{Diploma con ``Mención de Honor''}{}{}{Res. Rectoral 1243/14.}
\cventry{2008}{Premio a la excelencia académica}{Banco Roela}{Otorgado al mejor alumno graduado de la escuela secundari}{}{}
\cventry{2008}{Medalla de mérito académico}{Instituto Secundario Dr. Manuel Lucero}{Otorgado al mejor alumno del último año de la escuela secundaria}{}{}

%----------------------------------------------------------------------------------------
%	OTHER STUFF SECTION
%----------------------------------------------------------------------------------------
\section{Otros reconocimientos}
\cventry{2013}{Abanderado de la Facultad de Ciencias Químicas}{Universidad Nacional de Córdoba, año 2013}{}{}{Res. 726/13 H.C.D.}
\cventry{2013}{Primer escolta de la bandera nacional de la Universidad Nacional de Córdoba}{}{}{}{Res. Rectoral 1439/13}
\cventry{2012}{Segundo escolta de la bandera nacional de la Facultad de Ciencias Químicas}{Universidad Nacional de Córdoba}{}{}{Res. 688/12 H.C.D.}
\cventry{2009}{Preselección para la Olimpíada Internacional de Química 2009}{Entrenamiento teórico práctico de 2 meses en la FCEN, UBA}{Preparación para la IChO 2009 (Reino Unido)}{}{}
\cventry{2007--2008}{Olimpíada Argentina de Química}{Universidad de Buenos Aires}{Medalla de oro en nivel 1 (2007) y nivel 2 (2008). Mejor examen regional (2007 y 2008).}{}{}
\cventry{2006--2008}{Feria de Ciencia y Tecnología}{Agencia Córdoba Ciencia y Ministerio de Ciencia y Tecnología}{Diploma de mención especial (2006). 2do. lugar en etapa provincial (2007). Participación en instancia nacional, Ciudad Autónoma de Buenos Aires (2007)}{}{}
\cventry{2006--2008}{Olimpíada de Matemática Argentina}{Diploma de mención especial en instancia provincial (2008). Promoción a la instancia nacional (2007)}{}{}{}


\end{document}